% LaTeX Code for 8086 Assembly Language Basics
\documentclass[a4paper,12pt]{article}
\usepackage[utf8]{inputenc}
\usepackage{amsmath}
\usepackage{hyperref}

\title{8086 Assembly Language Basics}
\author{}
\date{}

\begin{document}

\maketitle

\section*{Registers Overview}

\subsection*{General Purpose Registers:}
\begin{itemize}
    \item \textbf{AX (Accumulator Register):}
    \begin{itemize}
        \item Primary register for arithmetic, logic, and data transfer operations.
        \item Divided into \textbf{AH} (High byte) and \textbf{AL} (Low byte).
    \end{itemize}

    \item \textbf{BX (Base Register):}
    \begin{itemize}
        \item Used to hold memory addresses for addressing data.
        \item Divided into \textbf{BH} (High byte) and \textbf{BL} (Low byte).
    \end{itemize}

    \item \textbf{CX (Count Register):}
    \begin{itemize}
        \item Primarily used as a loop counter.
        \item Divided into \textbf{CH} (High byte) and \textbf{CL} (Low byte).
    \end{itemize}

    \item \textbf{DX (Data Register):}
    \begin{itemize}
        \item Used for I/O operations and extended arithmetic operations.
        \item Divided into \textbf{DH} (High byte) and \textbf{DL} (Low byte).
    \end{itemize}
\end{itemize}

\subsection*{Pointer and Index Registers:}
\begin{itemize}
    \item \textbf{SI (Source Index):} Used for source addressing in string operations.
    \item \textbf{DI (Destination Index):} Used for destination addressing in string operations.
    \item \textbf{BP (Base Pointer):} Points to data on the stack.
    \item \textbf{SP (Stack Pointer):} Points to the top of the stack.
\end{itemize}

\section*{Important Concepts}

\subsection*{DB}
\begin{itemize}
    \item \textbf{Define Byte:} A directive to allocate memory and initialize it with byte-sized data.
\end{itemize}
\begin{verbatim}
msg DB "Hello, 8086!", 0Dh, 0Ah, "$"
num DB 100
\end{verbatim}
- \texttt{msg} stores a string ending with \texttt{\$} (used by DOS interrupts).
- \texttt{num} allocates a single byte initialized to 100.

\subsection*{Labels}
\begin{itemize}
    \item A marker in the code that acts like a bookmark for loops or jumps.
\end{itemize}
\begin{verbatim}
print_loop:
    ; Code here
\end{verbatim}

\subsection*{MOV Instruction}
\begin{itemize}
    \item Transfers data between registers or between memory and registers.
\end{itemize}
\begin{verbatim}
MOV AX, BX  ; Copy the value of BX into AX
MOV AL, 5   ; Load 5 into AL (lower byte of AX)
\end{verbatim}

\section*{Code Example}

\subsection*{Complete Program:}
\begin{verbatim}
.model small
.stack 100h
.data
    msg DB "Hello, 8086!", 0Dh, 0Ah, "$"   ; Message to print
.code
main proc
    ; AX: Arithmetic example
    MOV AX, 5         ; Load 5 into AX
    ADD AX, 10        ; Add 10 to AX (AX = 15)

    ; BX: Addressing example
    MOV BX, OFFSET msg  ; Load address of 'msg' into BX

    ; CX: Loop counter
    MOV CX, 5         ; Set loop counter to 5
print_loop:
    MOV AH, 2         ; Function to print a character
    MOV DL, '*'       ; Load '*' into DL
    INT 21h           ; Print the character
    LOOP print_loop   ; Decrement CX and jump if CX > 0

    ; DX: I/O operation
    MOV AH, 9         ; Function to print a string
    MOV DX, BX        ; Load address of 'msg' into DX
    INT 21h           ; Print the string

    HLT               ; Halt execution
main endp
end main
\end{verbatim}

\subsection*{Code Explanation:}
\begin{enumerate}
    \item \textbf{.model small:} Specifies the memory model (small = single code and data segment).
    \item \textbf{.stack 100h:} Reserves 256 bytes (100h) for the stack.
    \item \textbf{.data:} Segment for declaring variables and data.
    \item \textbf{msg DB \"Hello, 8086!\", 0Dh, 0Ah, \"\$\":} Defines a string ending with \texttt{\$}, used by \texttt{INT 21h} to print strings.
    \item \textbf{MOV AX, 5:} Loads the value 5 into the AX register.
    \item \textbf{ADD AX, 10:} Adds 10 to the value in AX (AX becomes 15).
    \item \textbf{MOV BX, OFFSET msg:} Stores the memory address of \texttt{msg} into the BX register.
    \item \textbf{MOV CX, 5:} Initializes the loop counter to 5.
    \item \textbf{print\_loop:} Label marking the start of the loop.
    \item \textbf{MOV AH, 2:} Prepares for a single-character print operation.
    \item \textbf{MOV DL, '*':} Loads the ASCII value of \texttt{*} into DL.
    \item \textbf{INT 21h:} DOS interrupt for performing I/O operations.
    \item \textbf{LOOP print\_loop\:} Decrements CX and jumps to \texttt{print\_loop} if CX > 0.
    \item \textbf{MOV AH, 9:} Prepares for a string print operation.
    \item \textbf{MOV DX, BX:} Loads the address of \texttt{msg} (stored in BX) into DX.
    \item \textbf{HLT:} Halts the program.
\end{enumerate}

\section*{Quick Revision (Bullet Notes)}
\begin{itemize}
    \item \textbf{AX:} Arithmetic and data transfer.
    \item \textbf{BX:} Memory addressing.
    \item \textbf{CX:} Loop counter.
    \item \textbf{DX:} I/O operations.
    \item \textbf{DB:} Define byte-sized variables or strings.
    \item \textbf{Labels:} Used for loops or jumps.
    \item \textbf{MOV Instruction:} Transfers data between registers/memory.
    \item \textbf{INT 21h:} DOS interrupt for I/O.
    \begin{itemize}
        \item AH = 2: Print a single character (character in DL).
        \item AH = 9: Print a string (address in DX).
    \end{itemize}
    \item \textbf{HLT:} Stops execution.
\end{itemize}
\newpage
\section*{Topics Covered}

Today, we explored more advanced concepts in assembly language programming, focusing on conditional branching, printing values, and structured control flow. Below is a detailed summary of the key topics:

\section{Conditional Branching}
\begin{itemize}
    \item In assembly, conditional branching is handled using instructions like \texttt{CMP} (compare) and conditional jumps such as:
    \begin{itemize}
        \item \texttt{JE} (Jump if Equal)
        \item \texttt{JG} (Jump if Greater)
        \item \texttt{JL} (Jump if Less)
        \item \texttt{JNE} (Jump if Not Equal)
    \end{itemize}
    \item Example of a simple conditional branch:
    \begin{verbatim}
    CMP AX, BX
    JG GREATER
    ; Code for else block
    JMP END_IF
    GREATER:
    ; Code for greater block
    END_IF:
    \end{verbatim}
\end{itemize}

\section{Printing Values}
\begin{itemize}
    \item To print data, we use the \texttt{INT 21H} interrupt with specific function codes:
    \begin{itemize}
        \item \texttt{AH = 09H} to display a string. The string must be terminated by a \texttt{\$} symbol.
        \item \texttt{AH = 02H} to display a single character (with the character stored in \texttt{DL}).
    \end{itemize}
    \item Example of printing a string:
    \begin{verbatim}
    LEA DX, STRING
    MOV AH, 09H
    INT 21H
    \end{verbatim}
    \item Example of printing a single character:
    \begin{verbatim}
    MOV DL, 'A'
    MOV AH, 02H
    INT 21H
    \end{verbatim}
\end{itemize}

\section{Structured Control Flow}
\begin{itemize}
    \item Assembly does not have built-in blocks for conditional or nested control flow.
    \item Explicit labels and jumps are used to define the structure of conditions and loops.
    \item Example of a nested condition:
    \begin{verbatim}
    CMP AX, BX
    JG GREATER
    CMP CX, DX
    JE EQUAL
    ; Else-Else block
    JMP END_NESTED
    EQUAL:
    ; Else-If block
    JMP END_NESTED
    GREATER:
    ; If block
    END_NESTED:
    ; Code continues
    \end{verbatim}
\end{itemize}

\section{LEA Instruction}
\begin{itemize}
    \item \texttt{LEA} (Load Effective Address) is used to load the address of a variable into a register, typically \texttt{DX}.
    \item Example:
    \begin{verbatim}
    LEA DX, STRING
    \end{verbatim}
    This moves the address of \texttt{STRING} into the \texttt{DX} register.
\end{itemize}

\section{Complete Example: Nested Conditions}
\begin{verbatim}
.MODEL SMALL
.STACK 100H

.DATA
    NUM1 DW 7
    NUM2 DW 5
    NUM3 DW 9
    NUM1_MSG DB 'NUM1 is the greatest$', 0
    NUM2_MSG DB 'NUM2 is the greatest$', 0
    NUM3_MSG DB 'NUM3 is the greatest$', 0

.CODE
MAIN PROC
    MOV AX, @DATA
    MOV DS, AX

    ; Compare NUM1 and NUM2
    MOV AX, NUM1
    CMP AX, NUM2
    JG CHECK_NUM3           ; If NUM1 > NUM2, check against NUM3

    ; Else, compare NUM2 and NUM3
    MOV AX, NUM2
    CMP AX, NUM3
    JG NUM2_IS_GREATEST     ; If NUM2 > NUM3, NUM2 is greatest

    ; Else, NUM3 is greatest
    MOV AH, 09H
    LEA DX, NUM3_MSG
    INT 21H
    JMP END_PROGRAM

NUM2_IS_GREATEST:
    ; Code for NUM2 > NUM3
    MOV AH, 09H
    LEA DX, NUM2_MSG
    INT 21H
    JMP END_PROGRAM

CHECK_NUM3:
    ; Compare NUM1 and NUM3
    CMP AX, NUM3
    JG NUM1_IS_GREATEST     ; If NUM1 > NUM3, NUM1 is greatest

    ; Else, NUM3 is greatest
    MOV AH, 09H
    LEA DX, NUM3_MSG
    INT 21H
    JMP END_PROGRAM

NUM1_IS_GREATEST:
    ; Code for NUM1 > NUM2 and NUM3
    MOV AH, 09H
    LEA DX, NUM1_MSG
    INT 21H

END_PROGRAM:
    MOV AX, 4C00H
    INT 21H
MAIN ENDP
END MAIN
\end{verbatim}

\end{document}